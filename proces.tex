\chapter{Het te volgen proces van de BP}

In de hoofdstuk wordt kor het proces geschetst ter indiening van uw bachelorproef. 

\section{Bachelorproefvoorstel uitschrijven}
Als eerste stap schrijf je een bachelorproefvoorstel uit. Hierin vermeld je zeker volgende elementen:

\begin{description}
	\item [Context]:  Waarom is dit werk belangrijk?
	\item [Nood] :  Waarom moet dit onderzocht worden?
	\item [Taak]:  Wat ga je (ongeveer) doen?
	\item [Object]: Wat staat in dit document geschreven?
	\item [Resultaat]: Wat verwacht je van je onderzoek?
	\item [Conclusie]: Wat verwacht je van van de conclusies?
	\item [Perspectief]: Wat zegt de toekomst voor dit werk?
\end{description}

De deadlines voor het indienen zijn de volgende:
\begin{framed}
	\begin{itemize}
		\item Semester 1 : \voorstelSemEen{}
		\item Semester 2 : \voorstelSemTwee{}
	\end{itemize}
\end{framed}

Indienen doe je op Chamilo onder opdrachten, BP voorstel.

\begin{framed}
	Het is verplicht de voorziene template (zie Chamilo - links) te gebruiken voor het uitschrijven van uw bachelorproefvoorstel.
\end{framed}

Wil je meer info rond het vinden van een onderwerp of het uitschrijven van het voorstel, zie dan hoofdstuk \ref{ch:onderwerp} en de cursus onderzoekstechnieken van het tweede jaar.

\section{Feedback op BPVoorstel}
Uw promotor wordt toegewezen op basis van de inhoud van uw voorstel en voorziet de nodige feedback. Die feedback komt op Chamilo te staan, onder de rubriek feedback van uw ingediende voorstel. De bedoeling is dat u de feedback verwerkt in uw voorstel en opnieuw voorlegt aan de toegekende promotor.

De deadlines voor het ontvangen van de feedback zijn de volgende:
\begin{framed}
\begin{itemize}
	\item Semester 1: \feedbackVoorstelSemEen{}
	\item Semester 2: \feedbackVoorstelSemTwee{}
\end{itemize}
\end{framed}

\section{Co-promotor}
Copromotoren zijn onderzoekers, docenten of mensen uit het werkveld met een bijzondere expertise op het vakgebied van uw bachelorproef. De copromotor is degene die de feitelijke en inhoudelijke begeleiding van de bachelorproef op zich neemt, terwijl de promotor zorgt voor een meer formele begeleiding en de theoretische aspecten van het onderzoek in de gaten houdt. 

\begin{framed}
Het is verplicht een co-promotor te zoeken en die in contact te laten treden met uw promotor. 

Het spreekt voor zich dat familie van de onderzoeker niet de taak van co-promotor kan opnemen.
\end{framed}

\section{Uitwerken van de bachelorproef}
Nadat de feedback verwerkt is in uw bachelorproefvoorstel ga je aan de slag. De volgende hoofdstukken in dit document helpen je bij het afbakenen van uw onderzoek, het opzetten van experimenten, uw literatuuronderzoek enz. Ga ook  nog eens door de cursus onderzoekstechnieken, want uw analyses moeten gemotiveerd en met de nodige statistiek beschreven worden.

\section{Indienen van de Bachelorproef}
Nadat uw bachelorproef opgesteld is volgens de richtlijnen van de template en de voorwaarden zoals ze beschreven staan in dit document, dient u in.
\begin{enumerate}
	\item Je dient de pdf versie in op Chamilo, opdrachten onder de juiste rubriek 
	\item Je dient de pdf versie in op de bachelorproefwebsite, die je kan terugvinden bij links op Chamilo 
	\item Je stuurt de pdf door via email naar uw co-promotor. 
\end{enumerate}

De deadlines hiervoor zijn de volgende. 
\begin{framed}
\begin{itemize}
	\item Semester 1: \indienenBPSemEen{}
	\item Semester 2:\indienenBPSemTwee{}
	\item Herexamen: \indienenBPderdeeEx{}
\end{itemize}
\end{framed}
\subsection{Papieren versie}
Indien uw promotor en uw co-promotor een papieren versie wensen, dan spreek je zelf af met de promotor en de co-promotor hoe deze bezorgd dienen te worden. Indien er geen papieren versie gevraagd wordt, dan hoef je geen verdere stappen te ondernemen. 

\subsection{Presentatie}
Nadat uw bachelorproef ontvankelijk verklaard is door uw promotor (en dus alle voorwaarden vervuld zijn) wordt u toegelaten tot de verdediging. De verdediging bestaat uit 3 delen.
\begin{enumerate}
	\item U krijgt 20 minuten om uw resultaten te presenteren
	\item Daarna worden 20 minuten voorzien om de juryleden vragen te laten stellen
	\item Daarna mag u het lokaal verlaten en wordt er door de juryleden gedelibereerd
\end{enumerate}
De uitslag wordt u bekend gemaakt bij de bekendmaking van alle andere punten.

