\chapter{Het te volgen proces van de BP}

In de hoofdstuk wordt kort het proces geschetst ter indiening van uw bachelorproef. 




\section{Bachelorproefvoorstel uitschrijven}
\label{sec:bpuitschrijven}
Als eerste stap schrijf je een bachelorproefvoorstel uit. Hierin vermeld je zeker volgende elementen:

\begin{description}
	\item [Context]:  Waarom is dit werk belangrijk?
	\item [Nood] :  Waarom moet dit onderzocht worden?
	\item [Taak]:  Wat ga je (ongeveer) doen?
	\item [Object]: Wat staat in dit document geschreven?
	\item [Resultaat]: Wat verwacht je van je onderzoek?
	\item [Conclusie]: Wat verwacht je van van de conclusies?
	\item [Perspectief]: Wat zegt de toekomst voor dit werk?
\end{description}

De deadlines voor het indienen zijn de volgende:
\begin{framed}
	\begin{itemize}
		\item Semester 1 : \voorstelSemEen{}
		\item Semester 2 : \voorstelSemTwee{}
	\end{itemize}
\end{framed}

Indienen doe je op Chamilo onder opdrachten, BP voorstel.

\begin{framed}
	Het is verplicht de voorziene template (zie Chamilo - links) te gebruiken voor het uitschrijven van uw bachelorproefvoorstel. Alle stijlelementen \textbf{moeten} gebruikt worden.
\end{framed}

Wil je meer info rond het vinden van een onderwerp of het uitschrijven van het voorstel, zie dan hoofdstuk \ref{ch:onderwerp} en de cursus onderzoekstechnieken van het tweede jaar.

\section{Feedback op BPVoorstel}
Uw promotor wordt toegewezen op basis van de inhoud van uw voorstel en voorziet de nodige feedback. Die feedback komt op Chamilo te staan, onder de rubriek feedback van uw ingediende voorstel. De bedoeling is dat u de feedback verwerkt in uw voorstel en opnieuw voorlegt aan de toegekende promotor.

De deadlines voor het ontvangen van de feedback zijn de volgende:
\begin{framed}
\begin{itemize}
	\item Semester 1: \feedbackVoorstelSemEen{}
	\item Semester 2: \feedbackVoorstelSemTwee{}
\end{itemize}
\end{framed}

\section{Co-promotor}
De taken van de co-promotor staan beschreven in sectie \ref{sec:copromotor}.


\begin{framed}
Het is verplicht een co-promotor te zoeken en die in contact te laten treden met uw promotor. 

Het spreekt voor zich dat familie \& vrienden van de onderzoeker niet de taak van co-promotor kunnen opnemen.
\end{framed}

\section{Uitwerken van de bachelorproef}
Nadat de feedback verwerkt is in uw bachelorproefvoorstel ga je aan de slag. De volgende hoofdstukken in dit document helpen je bij het afbakenen van uw onderzoek, het opzetten van experimenten, uw literatuuronderzoek enz. Ga ook  nog eens door de cursus onderzoekstechnieken, want uw analyses moeten gemotiveerd en met de nodige \textbf{relevantie statistiek} beschreven worden.

\section{Indienen van de Bachelorproef}
Nadat uw bachelorproef opgesteld is volgens de richtlijnen van de template en de voorwaarden zoals ze beschreven staan in dit document, dient u in. De procedure geldt voor alle zittijden.
\begin{enumerate}
	\item Je dient de pdf versie in op Chamilo, opdrachten onder de juiste rubriek 
	\item Je dient de pdf versie in op de bachelorproefwebsite, die je kan terugvinden bij links op Chamilo 
	\item Je stuurt de pdf door via email naar uw co-promotor. 
\end{enumerate}

De deadlines hiervoor zijn de volgende. 
\begin{framed}
\begin{itemize}
	\item Semester 1: \indienenBPSemEen{}
	\item Semester 2: \indienenBPSemTwee{}
	\item Herexamen: \indienenBPderdeeEx{}
\end{itemize}
\end{framed}
\subsection{Papieren versie}
Indien uw promotor en uw co-promotor een papieren versie wensen, dan spreek je zelf af met de promotor en de co-promotor hoe deze bezorgd dienen te worden. Indien er geen papieren versie gevraagd wordt, dan hoef je geen verdere stappen te ondernemen. Je hoeft dus niet per definitie een papieren versie af te geven op het studentensecretariaat.

\subsection{Presentatie}
Nadat uw bachelorproef ontvankelijk verklaard is door uw promotor (en dus alle voorwaarden vervuld zijn) wordt u toegelaten tot de verdediging. De verdediging bestaat uit 3 delen.
\begin{enumerate}
	\item U krijgt 20 minuten om uw resultaten te presenteren
	\item Daarna worden 20 minuten voorzien om de juryleden vragen te laten stellen
	\item Daarna mag u het lokaal verlaten en wordt er door de juryleden gedelibereerd
\end{enumerate}
De uitslag wordt u bekend gemaakt bij de bekendmaking van alle andere punten.

\section{Derde examenperiode}
Als je niet geslaagd bent voor de Bachelorproef in de eerste of tweede examenperiode, word je doorverwezen naar de derde examenperiode (2e zittijd). Je promotor blijft in dat geval dezelfde. Het is belangrijk om je promotor op te zoeken op de feedback. Bespreek met haar/hem wat je voor de tweede zittijd gaat doen, volgens het leesverslag dat opgesteld is. Als student heb je recht op dit leesverslag. 

 Zorg er voor dat alle afspraken ivm.   onderwerp,  co-promotor en  opvolging   gemaakt  zijn voor de aanvang van het zomerreces! Tijdens het zomerreces zijn alle lectoren van de opleiding onbereikbaar. Ze hoeven dan ook niet op emails te antwoorden. Als je niet tijdig contact opgenomen hebt met je promotor, word je uitgesloten van de tweede zittijd. Je kan bij je onderwerp blijven, maar dat dan beter uitwerken.  Je bespreekt dan op de feedback met je promotor over hoe je je bachelorproef kan verbeteren. Bij het per email doorsturen van je finale bachelorproef naar je promotor, verwachten we ook een toelichting bij wat je precies veranderd hebt aan de bachelorproef t.o.v. de versie die je in de eerste zittijd hebt ingediend.

Als je dat wil, kan je ook een nieuw onderwerp kiezen.  In dat geval moet je opnieuw een voorstel uitwerken zoals tijdens de eerste zittijd en zo nodig een nieuwe co-promotor zoeken. Stuur je voorstel al vóór de feedback door naar je promotor zodat dit kan besproken worden. Je onderwerp moet goedgekeurd zijn voordat het zomerreces begint, zo niet kan je niet deelnemen aan de tweede zittijd.

