\chapter{Ontvankelijkheid en Evaluatie}
\section{Ontvankelijkheid}
Hieronder sommen we de voorwaarden op waaraan een bachelorproef moet voldoen om beoordeeld te worden in de huidige examenperiode. Wanneer één of meerdere voorwaarden niet voldaan zijn, krijg je 0/20 en word je doorverwezen naar de volgende examenkans. Je geeft in dat geval geen eindpresentatie.

\begin{enumerate}
	\item Het onderwerp van de bachelorproef is formeel aanvaard door je promotor en eventuele latere wijzigingen van de focus of het onderwerp is gebeurd in onderling overleg met de promotor.
	\item De finale draft is ruim voor de deadline naar je promotor gestuurd, ten laatste 2 weken voor de deadline of de door je promotor opgelegde einddatum
	\item Alle documenten zijn ingediend volgens de opgelegde deadlines. 
	\item Minstens 30 A4-bladzijden (excl. bijlagen, broncode, ruwe resultaten \dots)
	\item Minstens 10.000 woorden (excl. bijlagen, broncode, ruwe resultaten \dots)
	\item De lay-out van de bachelorproef (bv. lettertype, paragraafstijl, bladspiegel) is \textbf{volledig } conform de voorziene template (incl. bijlagen).
	\item Het niet toegelaten wijzigingen aan te brengen aan de vormgeving (voorblad, paragraafstijl, bibliografie \dots ).
	\item Bevat minimaal volgende onderdelen:
	 \begin{enumerate}
	 	\item Samenvatting - Abstract
	 	\item Inhoudsopgave 	
	 	\item Woord vooraf
	 	\item Inleiding met duidelijke probleemstelling en onderzoeksvraag
	 	\item Conclusies \& besluit (aparte hoofdstukken)
	 	\item Referentielijst volgens de APA stijl
	 \end{enumerate}
 	\item De bachelorproef doorstaat de plagiaatcontrole.
 	\item Voor alle afbeeldingen die je niet zelf gecreëerd hebt, is er in het bijschrift een verwijzing  naar  de  bron  opgenomen,  en  die  is  opgenomen  in  de  referentielijst. Overnemen van afbeeldingen zonder bronvermelding wordt beschouwd als plagiaat. Logo’s overnemen kan ook niet: die vallen onder de merkenwetgeving en kunnen enkel gebruikt worden na expliciete toestemming van de rechthebbende.
 	\item Als je een bachelorproef voor de tweede/derde/\dots keer indient, zijn er fundamentele verbeteringen/aanvullingen t.o.v. de versie ingediend bij de vorige examenkans. Deze zijn door de student toegelicht bij het indienen.
 	\item Er is een grondige spelling- en taalcontrole uitgevoerd.  Een bachelorproef met verschillende taal- of zetfouten , of met stukken tekst die letterlijk uit Google Translate lijken te komen, worden als onontvankelijk verklaard. 
\end{enumerate}

\section{Evaluatie}
Als je bachelorproef ontvankelijk verklaard is, evalueren we die aan de hand van een aantal indicatoren, onderverdeeld in een viertal thema’s:
\begin{itemize}
	\item Vorm bachelorproef: lay-out, afbeeldingen, opmaak referentielijst, taalgebruik, enz. Indien je de \LaTeX{} template gebruikt, heb je hier weinig tot geen problemen mee.
	\item Inhoud bachelorproef: structuur van de tekst, meerwaarde voor het werkveld, enz.
	\item Onderzoeksproces: contact met je promotor, methodologie, grondigheid waarmee je het onderwerp hebt aangepakt, enz.
	\item Presentatie en verdediging
\end{itemize}

De juryleden die de bachelorproef lezen, maken daar een leesverslag over.  Enerzijds geven ze hun waardering over bepaalde aspecten van het werk zoals hierboven opgesomd. Daarnaast worden ook een aantal verbeterpunten geformuleerd en te behalen doelstellingen in geval van niet slagen. 

Daarnaast wordt ook het evaluatieformulier ingevuld dat als basis dient voor het examencijfer.

De juryleden bezorgen de twee verslagen aan de promotor die er rekening mee houdt bij de toewijzing van het examencijfer. Tijdens de besloten beraadslaging komen de aanwezige juryleden tot een consensus over de waardering van de hierboven opgesomde criteria en dit wordt meegenomen in de beoordeling aan de hand van het evaluatiekader. Wanneer er geen consensus is, neemt de promotor de eindbeslissing, rekening houdend met de verschillende meningen en met de verwachtingen vanuit de opleiding over de kwaliteit vaneen bachelorproef.

De opgestelde documenten worden bezorg aan de bachelorproefcoödinator door ze in te dienen op Chamilo.


