\chapter{Wie is wie}

In dit hoofdstuk wordt beschreven wie wat doet bij het volledige bachelorproeftraject.
\section{De student of onderzoeker}
Van de onderzoeker worden volgende elementen verwacht:
\begin{itemize}
	\item Zelf een onderwerp kiezen en uitwerken, of één kiezen uit de lijst van onderwerpen aangeboden door externen. Indien je zelf een onderwerp gekozen hebt, moet je een geschikte co-promotor zoeken die je inhoudelijk kan bijstaan. \textbf{Dit is verplicht.} Indien het onderwerp uit de opgegeven lijst komt, afspreken met de opdrachtgever (co-promotor) om het te bespreken.
	\item Het gekozen onderwerp verder uitwerken (werktitel, context, probleemstelling en onderzoeksvraag,  methodologie,  planning en referentielijst)   en het voorstel indienen (zie Sectie\ref{sec:bpuitschrijven}).
	\item Zelfstandig op zoek gaan naar de nodige informatie om je in te werken in hetvakdomein (vakliteratuur, handleidingen, lezingen op conferenties, enz.)
	\item Op een deontologisch verantwoorde manier omgaan met de gegevens en middelen die je ter beschikking krijgt van de opdrachtgever of de Hogeschool.
	\item Afspraken met promotor en co-promotor i.v.m. opvolging en tussentijdse rapportering maken en naleven.
	\item De bachelorproef schrijven.
\end{itemize}
We verwachten dat je zelf het initiatief neemt om je promotor te contacteren wanneer dat nodig is. Het is jouw verantwoordelijkheid om je promotor regelmatig op de hoogte te brengen van waar je staat en tussentijdse feedback te vragen over wat je tot dan toegeschreven/gerealiseerd hebt. Je spreekt met je promotor af hoe vaak je over je vooruitgang rapporteert, welke tussentijdse deliverables je aflevert en wat precies de verwachtingen daarrond zijn. 

\textbf{Volledig op eigen houtje werken en een bachelorproef indienen die je promotor nooit op voorhand heeft kunnen inzien wordt niet aanvaard.}

\subsection{Verzekering}
Er gelden speciale regels voor studenten die enkel hun bachelorproef (BP) moeten hernemen. Er moet dan rekening mee worden gehouden dat  de stageovereenkomst en het bijhorend stagereglement hier niet meer van toepassing zijn omdat het uiteraard niet langer om een stage gaat. Met name 
\begin{itemize}
	\item voor studenten die stage en BP combineren maar die na afloop van de stage nog bij de onderneming of de instelling of in de organisatie aanwezig moeten zijn in het kader van de BP (bijv. bij 2e examenkans).
	\item voor studenten die stage en BP in een verschillend semester opnemen.
\end{itemize}
\begin{framed}
Indien dit het geval is wordt verwacht dat de student samen met zijn promotor de ``Overeenkomst Bachelorproef'' invult. Dit document vind je terug op Chamilo. Gelieve hierbij ook rekening te houden dat de hogeschool vooraf op de hoogte moet zijn van het tijdstip, de locatie, de student en de duur en dat hiervoor vooraf goedkeuring moet verleend zijn. Daarom moet de Bachelorproefcoördinator op tijd op de hoogte gebracht worden.
\end{framed}


\section{Promotor}
De promotor is een lector van de opleiding die de bachelorproef opvolgt en de kwaliteit ervan borgt. De promotor is niet noodzakelijk een expert in het vakdomein, maar observeert vooral het onderzoeksproces en begeleidt de student daar in. De taken van de promotor omvatten:

\begin{itemize}
	\item Het beoordelen van het voorstel van het bachelorproefonderwerp, er feedback opgeven en bijsturen tot het voldoet aan de aanvaardingscriteria
	\item Tussentijdse opvolging van de vooruitgang van de student.
	\item Opleggen  van  deadlines  voor  opleveren  van  deelresultaten
	\item Feedback geven en zo nodig bijsturen op het vlak van methodologie, planning en organisatie
	\item Feedback geven op een finale draft van de bachelorproef wat betreft opmaak, opbouwen inhoud. Waarborgen dat de opgelegde template gevolgd wordt.
	\item Beoordelen of een ingediende bachelorproef ontvankelijk is voor verdediging binnen de huidige examenperiode en overleggen met de bachelorproefcoördinator indien dit niet het geval is.
	\item Deelnemen aan de jurering
	\item Een leesverslag opstellen en op vraag van de student bezorgen na bekendmaking van het examencijfer
	\item Het opstellen van het evaluatiedocument volgens de evaluatierichtlijnen. Dit is niet opvraagbaar door de student.
	\item Als jurylid fungeren voor andere bachelorproeven
	\item Schriftelijke versie bachelorproef, evaluatieschema en juryverslagen archiveren	
\end{itemize}

\section{Co-promotor}
\label{sec:copromotor}
Copromotoren zijn onderzoekers, docenten of mensen uit het werkveld met een bijzondere expertise op het vakgebied van uw bachelorproef. De copromotor is degene die de feitelijke en inhoudelijke begeleiding van de bachelorproef op zich neemt, terwijl de promotor zorgt voor een meer formele begeleiding en de theoretische aspecten van het onderzoek in de gaten houdt. 

De taken van de co-promotor zijn: 

\begin{itemize}
	\item Meer uitleg geven over het aangeboden onderwerp
	\item Aanreiken van achtergrondinformatie: vakliteratuur over het onderwerp, besprekenvan requirements (indien van toepassing voor het onderwerp)
	\item Inhoudelijk feedback geven over uitgewerkt onderzoeksvoorstel en tussenresultaten
	\item Zetelen in de jury (leesjury en presentatie) en de bachelorproef evalueren a.h.v. een kort verslag 
	\item Co-promotoren worden ook uitgenodigd om in de jury van andere bachelorproeven te zetelen.
\end{itemize}

\section{Bachelorproefcoördinator}
De bachelorproefcoördinator organiseert het praktisch verloop van de bachelorproeven.Dat omvat:
\begin{itemize}
	\item Richtlijnen vastleggen en uitschrijven
	\item Verzamelen voorstellen onderwerpen uit het werkveld
	\item Promotoren toewijzen aan studenten
	\item Organisatie bachelorproefpresentaties: opstellen presentatierooster, lokaaltoewijzing,samenstelling jury, enz.
	\item Overleggen met promotoren in geval van mogelijk niet-slagen en studenten en desgevallend op de hoogte brengen van de beslissing, met motivatie
	\item Indien nodig tuchtprocedures opstarten en opvolgen
	\item Overlegmomenten beleggen met de promotoren i.v.m. organisatie en begeleidingvan de bachelorproef

\end{itemize}


