\chapter{Inleiding}
\label{sec:inleiding}


% TODO
% Doelstelling bachelorproef (uit studiefiche)
%
% Academische thesis vs. Bachelorproef, toegepast vs basisonderzoek
%
% Een scriptie is een afgesloten geheel. Alle informatie die nodig is om het onderwerp ten gronde te begrijpen moet er in zitten. Het mag dus niet nodig zijn om nog externe bronnen na te lezen voordat je de tekst kan begrijpen.
%
%
% Onderwerp ten gronde uitspitten
%
%
% Volgorde van werken
% - werkomgeving opzetten
% - onderwerp kiezen
% - literatuurstudie
% - verzamelen en analyseren van data
% - conclusie, samenvatting en voorwoord schrijven

Als je de studiefiche van het opleidingsonderdeel Bachelorproef van de opleiding toegepaste informatica leest, dan kan je lezen dat de eindcompetentie de volgende is.

\begin{framed}
\begin{center}
	De student kan bestaande en innovatieve (IT-)oplossingen op een methodologisch correcte manier kritisch onderzoeken, evalueren en (her)ontwerpen of optimaliseren
\end{center}
\end{framed}


Waar het op neerkomt is dat we verwachten dat een bachelor toegepaste informatica in staat is 
\begin{inparaenum}[(i)]
	\item zich in te werken in een domein waar zij/hij nog niet noodzakelijk al expert in is en 
	\item en met de verzamelde informatie een oplossing kan uitwerken voor een concreet probleem uit het werkveld.
\end{inparaenum}

We beogen geen academisch wetenschappelijke studie, maar wel een toegepast onderzoek met
\begin{inparaenum}[(i)]
	\item voldoende technische diepgang,
	\item met statistisch relevante resultaten en
	\item een meerwaarde biedt voor het gekozen vakgebied
\end{inparaenum}

De bachelorproef geeft je de kans je beter te profileren en je interessesfeer in de verf te zetten.  Hierdoor wordt extra inzet en ontplooiing, ook buiten de domeinen die in de opleiding aan bod komen, mee beloond.

\section{Reglementen van toepassing}
Hier volgt een overzicht van de reglementen en richtlijnen die van toepassing zijn op de bachelorproef. Deze zijn bindend en elke student is verondersteld ze te kennen en zich er aan te houden. We geven enkele artikels mee, maar er wordt verwacht dat je alle onderstaande documenten doorgenomen hebt.

\begin{enumerate}
	\item Het algemeen Onderwijs- en Examenreglement (“OER”) 2017-2018, met name
	\begin{enumerate}
		\item Artikel 2: Gedragsregels
		\item Artikel 3: Intellectuele eigendomsrechten
		\item Artikel 31: Opleidingsonderdeel stage, bachelorproef, werkplekleren, praktijkopleidingsonderdeel enz
		\item Artikel 37 t.e.m. 46
		\item Artikel 69 : Taal van de evaluatie
	\end{enumerate}
	\item Het facultair reglement FBO (“FOER”) 2017-2018, met name
	\begin{enumerate}
		\item Artikel 38: Naleven van deadlines bij opdrachten in het kader van niet-periodegebonden evaluaties en periodegebonden
		evaluaties
		\item Artikel 45: Jury bij mondelinge examens (OER art. 52)
	\end{enumerate}
	\item Het stagereglement van de faculteit
	\item De studiefiche van het opleidingsonderdeel Bachelorproef
	\item Alle aankondigingen die via de Chamilo-cursus voor de Bachelorproef gepubliceerd worden. Je wordt verondersteld deze gelezen te hebben.
\end{enumerate}


