\chapter{Inleiding}
\label{ch:inleiding}

Deze gids is tot stand gekomen vanuit de begeleiding van de bachelorproef van de opleiding toegepaste informatica aan Hogeschool Gent. De bedoeling is om onze studenten wat houvast te geven in hoe ze hieraan in de eerste plaats moeten beginnen, en bevat heel wat tips i.v.m. methodologie, werken met {\LaTeX} voor een professionele opmaak, enz.

Hoewel de inhoud gegroeid is vanuit onze specifieke aanpak naar de bachelorproef toe, hopen we dat die ook relevant is voor andere (informatica-)studenten die een bachelorproef, thesis, of onderzoeksrapport moeten schrijven.

De doelstelling van de bachelorproef binnen onze opleiding is dat de student kan aantonen in staat te zijn om een ``klant'' (in de eigen organisatie of extern) advies te geven over het in een bedrijfscontext toepassen van een nieuwe ICT-technologie, of over de keuze tussen verschillende nieuwe ICT-producten. Dat begint bij het verzamelen van relevante informatie op een methodologisch correcte en objectieve manier. Dat betekent enerzijds het opzoeken van bestaande kennis, die te vinden is in de vakliteratuur, of via experten of belanghebbenden. Anderzijds wordt tijdens een onderzoek ook nieuwe kennis gecreëerd, bijvoorbeeld via zelf opgezette experimenten of een proof-of-concept. Al die informatie moet vervolgens gestructureerd en geanalyseerd worden, in het geval van kwantitatieve gegevens op een statistisch verantwoorde manier. Tenslotte wordt over het resultaat gerapporteerd in een op zichzelf staande tekst, de bachelorproef zelf.

Waar het op neerkomt is dat we verwachten dat een bachelor toegepaste informatica in staat is zich in te werken in een domein waar zij/hij nog niet noodzakelijk al expert in is en met de verzamelde informatie een oplossing uit te werken voor een \emph{concreet} probleem uit het werkveld. We beogen geen academisch wetenschappelijke studie, maar wel een toegepast onderzoek met voldoende technische diepgang dat een meerwaarde biedt in het gekozen vakgebied.

\section{Soorten onderzoek}
\label{sec:soorten-onderzoek}

Een bachelorproef hoeft geen academische masterthesis proberen na te bootsen. De professionele bachelor heeft zijn eigen waarde het is dus perfect mogelijk om te excelleren binnen de specifieke eigenheid van dit profiel. Het grote verschil bestaat in de eerste plaats uit het soort onderzoek dat gevoerd wordt.

Bij \emph{fundamenteel onderzoek}, dat typisch aan universiteiten uitgevoerd wordt, ligt de nadruk op het uitbreiden van onze kennis in het algemeen. Binnen computerwetenschappen kan het bijvoorbeeld gaan over het ontwikkelen van nieuwe en/of efficiëntere algoritmen voor probleemoplossing. Of de resultaten van fundamenteel onderzoek ook onmiddellijk toepasbaar zijn is van secundair belang.

\emph{Toegepast onderzoek}, daarentegen, probeert een antwoord te formuleren op een concrete vraag uit het werkveld. De onderzoeker zal dan aan de hand van de kennis die er op dit moment beschikbaar is (en gepubliceerd door vakexperten), proberen die vraag te beantwoorden. Dat kan bijvoorbeeld gaan over hoe een nieuwe technologie concreet in een bedrijfscontext kan toegepast worden, een keuze maken tussen verschillende alternatieve producten of technologieën, een vooronderzoek voorafgaand aan het ontwikkelen van een applicatie, enz. Bij toegepast onderzoek is de doelgroep ook heel specifiek, bijvoorbeeld één enkel bedrijf. We merken dan ook dat de onderwerpen die aangebracht zijn door bedrijven ook het best uitgewerkt zijn en het vaakst leiden tot een goede bachelorproef.

\section{Structuur van deze gids}
\label{sec:structuur}

De rest van deze gids is als volgt gestructureerd:

Hoofdstuk~\ref{ch:voorbereiding} helpt bij het opzetten van een werkomgeving, meer bepaald {\LaTeX} en een versiebeheersysteem.

Hoofdstuk~\ref{ch:onderwerp} geeft tips bij het kiezen van een onderwerp en uitschrijven van een onderwerp.

Hoofdstuk~\ref{ch:literatuuronderzoek} gaat gedetailleerd in op het voeren van een literatuuronderzoek, het bijhouden van referenties in een bibliografische databank en hoe dit correct in een {\LaTeX}-document te verwerken.

Hoofdstuk~\ref{ch:onderzoeksmethoden} bespreekt een aantal onderzoeksmethoden die typisch gebruikt worden in een bachelorproef, o.a. de vergelijkende studie, of het opzetten van experimenten.

Hoofdstuk~\ref{ch:schrijven}, tenslotte bevat een aantal tips in verband met het schrijven van de tekst, en enkele {\LaTeX}-specifieke richtlijnen.
